La instalación de cable de fibra óptica es uno de los elementos más costosos en la implantación PON y la forma de proceder depende de diversos factores, incluido el coste, los derechos de paso, las normas legales, la estética, etc., y de si la fibra se instalará en instalaciones nuevas (instalación ‘greenfield’) o en una despliegue existente en rutas activas (superposición/sobre construcción). Se utilizan tres métodos básicos de instalación de cables:


\begin{itemize}
\item Enterramiento directo. Con este método el cable se coloca enterrado en contacto directo con el suelo; esto se hace excavando zanjas, arando o perforando.

\item Instalación de conductos. En este caso, el cable óptico se coloca dentro de una red de conductos subterráneos. Pese a que la instalación inicial de conductos es más cara que una instalación bajo tierra directa, su uso hace que sea mucho más fácil agregar o retirar cables.

\item Instalación aérea. Con este enfoque, el cable se instala normalmente en postes o torres sobre el suelo. Este tipo de instalación, normalmente utilizada para la sobre construcción, es por lo general más asequible que la instalación enterrada y no requiere maquinaria pesada. El cable óptico puede coserse a un cable fiador o pueden emplearse cables ópticos autosoportados.
\end{itemize}


Para áreas densamente pobladas con dificultades de derecho de paso, también hay disponibles varios métodos alternativos. Por ejemplo, el cable puede instalarse en ranuras que se hayan cortado en el pavimento o dentro de tubos de desagüe, tubos de alcantarillado u otro tipo de conductos ya existentes.