
\fontsize{13}{14}\selectfont


La fibra óptica monomodo, con su ancho de banda prácticamente ilimitado, es actualmente el medio de transmisión preferido en las redes de telecomunicación, de transporte y metropolitanas. La utilización de cable de fibra óptica (en lugar de cable de cobre) reduce significativamente los costes del equipo y de mantenimiento, a la vez que aumenta drásticamente la calidad del servicio (QoS); y, ahora más que nunca, muchos clientes corporativos tienen acceso a servicios de fibra óptica de punto a punto (P2P).

Los cables de fibra óptica se implantan ahora en la última milla: el segmento de la red que va desde la oficina central (CO) al abonado. Dado que, hasta hace poco, ese segmento se basaba normalmente en el cobre, los servicios de alta velocidad disponibles para los clientes residenciales y las empresas pequeñas se limitaban a líneas de abonados digitales genéricas (xDSL) y transmisiones coaxiales de fibra híbridas (HFC). La principal alternativa (transmisión inalámbrica con servicio de retransmisión directa (DBS)) requiere una antena y un transceptor. Por tanto, en el contexto actual, con su enorme demanda de ancho de banda y de servicios de mayor velocidad a distancias mayores, el transporte basado en cobre e inalámbrico presenta las siguientes carencias:
      

\begin{itemize}
\item Ancho de banda limitado.
\item Diferentes medios y equipos que requieren un mantenimiento específico.
\end{itemize}


Pese a que los cables de fibra óptica superan todas esas limitaciones, uno de los obstáculos en la provisión de servicios de fibra óptica directamente a los hogares y a las pequeñas empresas ha sido el elevado coste de conectar a cada abonado a la CO.

Para superar los problemas de costes, actores importantes de la industria crearon la organización de normalización Red de Acceso de Servicio Completo (Full-Service Access Network, FSAN), fundada para facilitar el desarrollo de especificaciones adecuadas de sistemas de equipos de redes de acceso.

La Unión Internacional de Telecomunicaciones (ITU-T) convirtió las especificaciones FSAN en recomendaciones. La especificación FSAN para redes ópticas pasivas (PONs) basadas en ATM se convirtió en una norma internacional en 1998 y fue adoptada por la ITU como recomendación G.983.1.

